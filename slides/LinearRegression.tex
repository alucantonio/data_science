\documentclass{beamer}
\usepackage{subfig}
\usepackage{amsmath}
\usepackage{bm}
\DeclareMathOperator*{\argmax}{arg\,max}
\DeclareMathOperator*{\argmin}{arg\,min}


\title{Linear Regression}
\author{Prof. Alessandro Lucantonio}
\institute{Aarhus University - Department of Mechanical and Production Engineering}
\date{?/?/2023}

\begin{document}
	
	\frame{\titlepage}
	
	\section{Linear regression with one variable}

	\begin{frame}
		\frametitle{Weight-Height example}
		Dataset: heights and weights of different people.
		
		Task: build a model that predict the height given the weight. 
		
		\begin{figure}
			\centering
			\includegraphics[scale=0.5]{images/linear_regression_data}
			\caption{Data plot}
		\end{figure}
	\end{frame}

	\begin{frame}
		\frametitle{A solution - Linear regression model}
		Some remarks on data.
		\begin{itemize}
			\item Regression problem (continuous output).
			\item Data with different order of magnitude.
		\end{itemize}
		A possible solution to this problem is represented by \textbf{linear regression} (LR).
		\begin{figure}
			\centering
			\includegraphics[scale=0.4]{images/linear_regression_fit}
			\caption{Trained model (in red)}
		\end{figure}
	\end{frame}

	\begin{frame}
		\frametitle{General ingredients}
		Notation:
		\begin{itemize}
			\item $x$: a data sample.
			\item $y$: the data target corresponding to $x$
			\item $N$: number of data.
		\end{itemize}
	
		\vspace{5 mm}
		
		Model/hypothesis: $h_{\bm{w}}(x) = w_1x + w_0$, where $\bm{w} = [w_0, w_1]$ is the vector of parameter that has to be learned.
		
		In our example, $x$ is the weight of a single sample and $h_{\bm{w}}(x)$ corresponds to the prediction of its height. 
		
		\vspace{5 mm}
		
		Usually the vector $\bm{w}$ is called \textbf{weights vector} and the set $\mathcal{H}:= \{h_{\bm{w}}| \bm{w} \in \mathbb{R}^2\}$ is called \textbf{hypothesis space}.
		
		\vspace{5 mm}
		
		How to learn $w$ from data?
		
	\end{frame}

	\begin{frame}
		\frametitle{Mean squared error (MSE)}
		
		
		Given a training sample $x_i$ and a model $h_{\bm{w}}$ we can predict the target computing $h_{\bm{w}}(x_i)$. To evaluate how good is the prediction we compute the error $(h_{\bm{w}}(x_i) - y_i)^2$.
		
		\vspace{5 mm}
		
		$(h_{\bm{w}}(x_i) - y_i)^2 \geq 0$ and $(h_{\bm{w}}(x_i) - y_i)^2 = 0$ if and only if $h_{\bm{w}}(x_i) = y_i$. The \textbf{mean squared error} (MSE) is:
		$$E({\bm{w}}) := \frac{1}{N} \sum_{i=1}^{N} (h_{\bm{w}}(x_i) - y_i)^2.$$
		
		To find the best model we minimize the training error, hence in this case the MSE.
		
		$$\bm{w} \in \argmin_{\tilde{\bm{w}} \in \mathbb{R}^2} E(\tilde{\bm{w}}).$$
		
	\end{frame}

	\begin{frame}
		\frametitle{$n$-dimensional LR}
		\textsl{Dataset samples}.
		
		Previous case: $x \in \mathbb{R}, y \in \mathbb{R}$.
		
		\vspace{1 mm}
		
		Now:  $\bm{x} \in \mathbb{R}^n, y \in \mathbb{R}$.
		
		Notation: $x^i_j$ is the $j$-th coordinate of the $i$-th sample.
		
		\vspace{5 mm}
		
		\textsl{Hypothesis}.
		
		Previous case: 
		\begin{equation*}
			h_{w}(x) = w_1x + w_0,
		\end{equation*}
		where $w = [w_0, w_1]$.
		
		\vspace{1 mm}
		
		Now: 
		\begin{align*}
			h_{\bm{w}}(\bm{x}) &= w_{n}x_n + w_{n-1}x_{n-1} + \dots + w_1 x_1 + w_0\\
			&= \sum_{i=0}^n w_i \tilde{x}_i = \bm{w}^T \tilde{\bm{x}},\\
		\end{align*}
		where $\bm{w} = [w_0, \dots, w_n]$ and $\tilde{\bm{x}} = [1, x_1, \dots, x_n]$.
		
	\end{frame}

	\begin{frame}
		\frametitle{n-dimensional LR}
		\textsl{MSE}.
		
		Previous case: 
		\begin{equation*}
			E(\bm{w}) = \frac{1}{N} \sum_{i=1}^{N} (h_{\bm{w}}(x_i) - y_i)^2.
		\end{equation*}
		
		\vspace{1 mm}
		
		Now: 
		\begin{align*}
			E(\bm{w}) &= \frac{1}{N} \sum_{i=1}^{N} (h_w(\bm{x}^i) - y^i)^2\\
			&= \frac{1}{N} (\mathsf{X} \bm{w} - \bm{y})^T (\mathsf{X}\bm{w} - \bm{y})\\
			&= \frac{1}{N} ||\mathsf{X}\bm{w} - \bm{y}||^2
		\end{align*}
		
		where
		\begin{equation*}
			\mathsf{X} := \begin{bmatrix}
				\tilde{\bm{x}}^1\\
				\vdots\\
				\tilde{\bm{x}}^N 
			\end{bmatrix} \quad \bm{y} = \begin{bmatrix}
			y_1 \\
			\vdots\\
			y_N
		\end{bmatrix}.
		\end{equation*}
	\end{frame}

	\begin{frame}
		\frametitle{Spot the minimum - Gradient descent}
		How to find $\bm{w} \in \argmin_{\tilde{\bm{w}} \in \mathbb{R}^2} E(\tilde{\bm{w}})$?
		
		Main idea:
		\begin{itemize}
			\item Start with a random $\bm{w}^0$.
			\item For $j \geq 0$, update $\bm{w}^{j+1} := \bm{w}^{j} + \bm{d}^j$, where $\bm{d}^j$ is such that
			\begin{equation*}
				E(\bm{w}^{j+1}) \leq E(\bm{w}^j) 
			\end{equation*}
		\end{itemize}
		Gradient descent: $\bm{d}^j = - \alpha \nabla E(\bm{w}^j)$. $\alpha$ is called \textbf{learning rate}.
	\end{frame}

\end{document}